\section{Ergänzungen zu den Systemanforderungen}

\subsection{Funktionale Anforderungen}
\label{subsec:af-tabelle}

In Tabelle \ref{figure:af} finden sich die funktionalen Anforderungen in tabellarischer Form, die in Abschnitt \ref{sec:funktionale-af} textuell ausgeführt sind. Die Tabelle ist nach Bereichen und absteigend nach Priorität geordnet. Für die Priorität werden die Ziffern \textit{1} (höchste Priorität), \textit{2} (mittlere Priorität) und \textit{3} (niedrigste Priorität) verwendet.  

\clearpage
\setlength{\hoffset}{-7mm}


\medskip
\begin{figure}[H]
  \begin{tabular}{ | l | l | l | l |}
    \hline
    \textbf{Bereich} & \textbf{Nr.} & \textbf{Beschreibung} & \textbf{Priorität}\\ \hline
    \hline
    \textbf{Outlines} & \textbf{FA100} & Erstellen & 1\\ \hline
    & \textbf{FA101} & Titel ändern & 1\\ \hline
    & \textbf{FA102} & Löschen & 2\\ \hline
    \hline
    \textbf{Zeilen} & \textbf{FA200} & Hinzufügen & 1\\ \hline
    & \textbf{FA201} & Navigieren mit Funktionstasten & 1\\ \hline
    & \textbf{FA202} & Inhalt bearbeiten & 1\\ \hline
    & \textbf{FA203} & Bei Verlassen automatisch speichern & 1\\ \hline
    & \textbf{FA204} & Ein- und ausrücken & 1\\ \hline
    & \textbf{FA205} & Blockweise ein- und ausrücken & 1\\ \hline
    & \textbf{FA206} & Warnung vor Datenverlust bei Schließen des Fensters & 2\\ \hline
    & \textbf{FA207} & Nach oben/unten verschieben & 2\\ \hline
    & \textbf{FA208} & Größe anpassen an Länge des Textes  & 2\\ \hline
    & \textbf{FA209} & Ein- und ausklappen  & 2\\ \hline
    & \textbf{FA210} & Einklappstatus lokal speichern  & 2\\ \hline
    & \textbf{FA211} & Revisionen speichern  & 3\\ \hline
    & \textbf{FA212} & Zwischen Revisionen wechseln  & 3\\ \hline
    & \textbf{FA213} & Kommentieren & 3\\ \hline
    & \textbf{FA214} & Löschen & 3\\ \hline
    \hline
    \textbf{Replikation} & \textbf{FA300} & Outlines zum Server replizieren & 1\\ \hline
    & \textbf{FA301} & Änderungen an Outlines zum Server replizieren & 1\\ \hline
    & \textbf{FA302} & Outlines von Anderen erhalten & 1\\ \hline
    & \textbf{FA303} & Änderungen an Outlines von Anderen erhalten & 1\\ \hline
    & \textbf{FA304} & Benachrichtigung bei Änderungen von Anderen & 1\\ \hline
    & \textbf{FA305} & Wiederaufnahme oder Hinweis bei Internetzugang & 2\\ \hline
    & \textbf{FA306} & Outlines veröffentlichen und auswählen & 3\\ \hline
    & \textbf{FA307} & Statusmeldung über Verbindungszustand & 3\\ \hline
    & \textbf{FA308} & An- und ausstellen & 3\\ \hline
    \hline
    \textbf{Konflikte} & \textbf{FA400} & Eine Konfliktart automatisch lösen & 1\\ \hline
    & \textbf{FA401} & Eine Konfliktart manuell lösen & 1\\ \hline
    & \textbf{FA402} & Kombination aus unterschiedlichen Konfliktarten lösen & 2\\ \hline
    & \textbf{FA403} & Konflikte zwischen mehr als zwei Repliken lösen & 2\\ \hline
    & \textbf{FA404} & Mehrere Konfliktarten automatisch lösen & 3\\ \hline
    & \textbf{FA405} & Mehrere Konfliktarten manuell lösen & 3\\ \hline
    \hline
  \end{tabular}
  \caption{Anforderungen an das System}
  \label{figure:af}
\end{figure}

\clearpage
\setlength{\hoffset}{0mm}
