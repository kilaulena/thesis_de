\chapter*{Abstract}

Auf heutigen Webbrowsern und mobilen Endgeräten können komplexe Anwendungen ausgeführt werden. Diese ermöglichen Zusammenarbeit und Datenaustausch zwischen BenutzerInnen. Bei Laptops und Mobilfunkgeräten kann keine kontinuierliche Internetverbindung vorausgesetzt werden. Um dieses Problem zu berücksichtigen kann Datenreplikation eingesetzt werden, wobei die Daten regelmäßig synchronisiert und dabei konsistent gehalten werden müssen. In dieser Arbeit wird die Konzeption und prototypische Erstellung einer JavaScript-Anwendung beschrieben, die mithilfe der dokumentenorientierten Datenbank CouchDB einen verteilten Gliederungseditor umsetzt. Mit einem Gliederungseditor können z. B. Gedanken oder Konzepte hierarchisch geordnet aufgeschrieben werden. Neben der Einordnung des zu erstellenden Systems und einer Analyse der in Frage kommenden Lösungsmöglichkeiten werden die verwendeten Technologien beschrieben. Genau vorgestellt werden dabei CouchDB mit der eingebauten Master-Master-Replikation und der Möglichkeit, eine komplexe Applikation ohne Middleware zu implementieren. Die erstellte Anwendung läuft lokal im Browser und ist dadurch auch offline benutzbar. Konflikte werden bei der Synchronisation vom System, teilweise mit Benutzerunterstützung, aufgelöst. In der Arbeit wird abschließend die Einsetzbarkeit von CouchDB für eine verteilte Anwendung und speziell für den gewählten Anwendungsfall beurteilt.