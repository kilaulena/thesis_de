\chapter{Bewertung und Ausblick}
\label{chap:fazit}

Für eine Beurteilung des Erfolgs wird das entwickelte System zunächst mit den in Kapitel \ref{chap:systemanforderungen} aufgestellten Anforderungen  verglichen. Anhand dessen werden das Ergebnis bewertet und bisher ungelöste Probleme aus der Implementierung diskutiert. Dann wird ein Ausblick auf Zukunftsperspektiven der verwendeten Technologien gegeben. Den Abschluss bilden Empfehlungen für die weitere Entwicklung bzw. Forschung.


\section{Bewertung des Ergebnisses}

Die insgesamt siebzehn aufgestellten Muss-Kriterien (s. Abschnitt \ref{subsec:muss}) konnten vollständig umgesetzt werden. Dagegen konnten von sechzehn Kann-Kriterien (s. Abschnitt \ref{subsec:kann}) nur vier implementiert werden. Aus Zeitgründen nicht umgesetzt wurden Verschieben, Löschen, Kommentierung und Versionierung von Zeilen im Gliederungseditor, gezieltes Replizieren von einzelnen Outlines und explizites Ein- und Ausstellen der Replikation. 

Für den produktiven Einsatz müsste vor allem die Konfliktbehandlung weiterentwickelt werden. Wie in Abschnitt \ref{subsec:otherconflicts-design} beschrieben führen Konflikte, die durch gleichzeitiges abweichendes Ein- oder Ausrücken von Zeilen entstehen, zu Fehlern beim Aufbau der Outlines. Werden Zeilen gleichzeitig in drei oder mehr Repliken verändert, führt dies ebenfalls zu Konflikten, die nicht korrekt abgefangen werden. Dies ist in erster Linie ein Problem der Darstellung: die Benutzeroberfläche für das manuelle Lösen von Konflikten ist, wie in Abschnitt \ref{subsec:writeconflict-implementierung} dargestellt wurde, nur für zwei unterschiedliche Versionen einer Zeile konzipiert. Auch wenn dies im produktiven Einsatz nicht häufig vorkäme, müsste die manuelle Zusammenführung einer höheren Anzahl von Zeilenversionen im Hinblick auf die Gestaltung der Benutzeroberfläche grundlegend anders angegangen werden.

Den nichtfunktionalen Anforderungen wurde weitestgehend entsprochen. Verbesserungswürdig ist der Quelltext in Bezug auf Einfachheit und Redundanzfreiheit: Um ein durchgehend nach der MVC-Architektur aufgebautes Programm zu erhalten, müssten die Funktionen zur Steuerung der Replikation in den objektorientierten Programmierstil umgewandelt werden. Dies wurde bereits bei der Konflikterkennung, -präsentierung und -bearbeitung umgesetzt. Den Anforderungen an die Benutzeroberfläche (vgl. Abschnitt \ref{subsec:gui-anf}) kommt die Anwendung nach. Jedoch reagiert die Benutzeroberfläche auf Benutzereingaben aufgrund des notwendigen Verbindungsaufbaus zwischen Frontend und Datenbank mitunter so verzögert, dass der Arbeitsfluss für wenige Sekunden unterbrochen wird. Es bleibt zu prüfen, ob dies mit einer Optimierung der Benutzeroberfläche behoben werden kann.

Trotz der beschriebenen Mängel konnte die in der Aufgabenstellung beschriebene Anwendung weitestgehend erfolgreich entworfen und umgesetzt werden. Die wichtigste Leistung der Arbeit besteht darin, dass mit dem entwickelten Programm ein bestimmtes Paradigma der Internetbenutzung verwirklicht wird, nämlich das der Peer-to-Peer-Kommunikation. Im Vergleich zu üblichen Client-Server-Anwendungen haben Benutzer bei der hier konzipierten Architektur mehr Kontrolle über ihre Daten. Gemeinsames Arbeiten kann umgesetzt werden, ohne auf eine kontinuierliche Netzwerkverbindung und ständig verfügbare Server angewiesen zu sein.


\section{Die Zukunft der eingesetzten Technologien}

Die Implementierung gestaltete sich insgesamt zeitaufwändiger als gedacht. Um die HTTP-API, Replikation, Konfliktbehandlung und Überwachung der Datenbank auf Änderungen passend anzuwenden, musste eine beträchtliche Grundlagenarbeit geleistet werden. Für die Umsetzung der eigentlichen Fachlogik verblieb vergleichsweise wenig Zeit. Jedoch werden die eingesetzten Technologien fortlaufend verbessert und neue Bibliotheken und Frameworks entwickelt, durch die der Arbeitsaufwand für ähnliche Projekte in Zukunft geringer ausfallen dürfte. 

So enthält die CouchDB-Version 1.0 einige Features, die die Verbesserung der umgesetzten Anwendung deutlich vereinfachen kann \cite{couch:whatsnew}. Ihre Veröffentlichung wird auf das Ende der Bearbeitungszeit dieser Arbeit fallen \cite{couch:release1.0}. Mit CouchDB in der Version 1.0 wird es beispielsweise möglich sein, Dokumente einzeln unter Angabe ihrer ID zu replizieren, und nicht mehr nur die Datenbank als Ganzes. So kann eine gezielte Replikation von Outlines einfacher erfolgen. Des Weiteren wird die Unterstützung für das Betriebssystem Windows verbessert, wodurch die Plattformunabhängigkeit der Anwendung wachsen wird. Für zukünftige Releases von CouchDB ist außerdem geplant, Sharding nativ zu unterstützen [Lehnardt, Jan, persönliche Mitteilung, 09.07.2010]. Damit kann das Aufsetzen von CouchDB-Lounge in Zukunft entfallen.

Unter den aktuellen Neuentwicklungen ist das Framework \textit{Evently} zu erwähnen \cite{evently:website}. Damit kann ähnlich wie mit Sammy das Routing einer Anwendung umgesetzt werden, es wurde jedoch explizit für eventbasierte Anwendungen mit CouchDB entwickelt. Evently stellt eine Verbindung zwischen CouchDB-Views, dem Changes-Feed, HTML-Templates und definierten JavaScript-Callbacks her und gibt eine Struktur für die Organisierung des Quelltexts vor. Gegenüber den in dieser Arbeit verwendeten Mitteln kann Evently deutliche Produktivitätsvorteile bieten.
  
  

\section{Empfehlungen für die Weiterentwicklung}

Die Entwicklung des Gliederungseditors kann ohne größere Hindernisse fortgesetzt werden. Durch die Implementierung der Baumstruktur können Löschen bzw. Verschieben einer Zeile einfach umgesetzt werden, indem die Zeile nicht mehr angezeigt bzw. innerhalb des Baums an ihrer neuen Position eingebaut wird. Spalten im Gliederungseditor können ebenfalls ohne Schwierigkeiten umgesetzt werden, indem den Zeilen mehrere Textareas zugewiesen werden. 

Ein weiteres Arbeitsfeld wäre die Umsetzung von Zugriffskontrolle und Benutzerverwaltung. Einzelne Outlines könnten vom Benutzer als öffentlich oder privat gekennzeichnet werden und dementsprechend für die Replikation freigegeben sein oder nicht. Anwendungen, die mit verteilten Daten auf mobilen Endgeräten arbeiten, stellen erhöhte Anforderungen an Sicherheitsvorkehrungen:   

\begin{quote}
Providing high availability and the ability to share data despite the weak connectivity of mobile computing raises the problem of trusting replicated data servers that may be corrupt. This is because servers must be run on portable computers, and these machines are less secure and thus less trustworthy than those traditionally used to run servers. [...] Portable machines are often left unattended in unsecured or poorly secured places, allowing attackers with physical access to modify the data and programs on such computers. \citelit[Kap. 1]{servercorruptness}
\end{quote}


Demzufolge wäre für einen Produktiveinsatz der Implementierung von Zugriffskontrolle ein hoher Stellenwert einzuräumen. Die Priorität für die Weiterentwicklung sollte nach Meinung der Autorin allerdings zuerst darin liegen, die Peer-to-Peer-Fähigkeiten der Anwendung weiter auszubauen. Instanzen der Anwendung könnten freigegebene Outlines über einen Webservice propagieren. Protokolle wie \textit{Bonjour} ermöglichen eine automatische Erkennung von Netzwerkdiensten in lokalen IP-Netzen \cite{bonjour:website}. Ein solches Protokoll könnte verwendet werden, damit Instanzen der Anwendung sich gegenseitig in einem Netzwerk erkennen und die Möglichkeit bieten können, Outlines direkt miteinander zu replizieren. Damit könnten Dokumente zB. in einem Büro oder auf einer Konferenz auch ohne Internetverbindung gleichzeitig bearbeitet werden.
  
Die Umsetzung der Aufgabenstellung kann als gelungen bezeichnet werden. Es wird allerdings noch beträchtlichen Entwicklungsaufwands bedürfen, bis die entstandene Anwendung für alle in Abschnitt \ref{subsec:einsatzmoegl} aufgezählten Einsatzmöglichkeiten produktiv eingesetzt werden kann. Wenn mehrere Benutzer ein Outline zeitgleich in größerem Umfang bearbeiten und erst nach zahlreichen Änderungen synchronisieren, werden die Anzahl und Komplexität der Konflikte von der Anwendung in der aktuellen Version noch nicht befriedigend und stabil bewältigt. Mit CouchDB und den anderen evaluierten Technologien kann ein Verteiltes System gut umgesetzt werden, das Mergen der Daten liegt jedoch im Aufgabenbereich der Anwendungsentwicklerin. Das CouchDB Entwicklerteam plant jedoch, vorgefertigte Lösungen für häufig vorkommende Konflikt-Szenarien anzubieten [Lehnardt, Jan, persönliche Mitteilung, 09.07.2010]. Nichtsdestoweniger ist das Mergen beim vorliegenden Anwendungsfall eine überdurchschnittliche Herausforderung, da durch den gewählten Lösungsansatz die Dokumente (die einzelnen Zeilen einer Outline) sehr granular gewählt und stark verknüpft sind. Einsatzgebiete, in denen seltener gleichzeitig an einem Dokument gearbeitet wird und Konflikte daher seltener auftreten, dürften mit deutlich weniger komplexen Lösungen auskommen. Denkbar sind Adressbücher, Kalender, Kundendaten, oder auch Dienste, in denen Nachrichten ausgetauscht werden. In \citelit[Kap. 10]{couchdb} und \cite{couch:whatsnew} finden sich weitere Anregungen.

